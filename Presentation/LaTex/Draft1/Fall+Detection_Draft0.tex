% LLNCS class package used for SMU Data Science Review Journal
\documentclass{llncs}

% Packages to be used in the document
\usepackage{graphicx} % display figures sample figures - EPS format (preferred), PDF/JPEG also acceptable
\usepackage{booktabs} % Better horizontal rules in tables
\usepackage{multirow} % Better combined rows in tables
\usepackage{xcolor} % Change font color for editing purposes
\usepackage{hyperref} % Footnote references
\usepackage{mathtools}



% Paper Title
\title{\textbf{Signal Processing of Geospatial and Biometric Data from Wearable Devices for Fall Detection}}

% Complete List of Authors with Affiliations
\author{Joseph Caguioa\inst{1}\and Andy Nguyen\inst{1}\and  Michael J. Wolfe\inst{1}\and Jacquelyn Cheun\inst{1}} %Capstone Advisor: Jacquelyn Cheun

\institute{Master of Science in Data Science, Southern Methodist University, Dallas TX 75275 USA
	      \email{\{jcaguioa, andynguyen, mwolfe\}@smu.edu} 
	      % \and Add further credentials/companies for yourselves @Michael/Joseph
	      }	  

% Begin Document 
\begin{document}

% Typeset the title & authors of paper
\maketitle
% Reset footnote counter
\setcounter{footnote}{0}

% Abstract Typeset
\begin{abstract}
In this paper, we present a detection algorithm that accurately differentiates the event of a person falling from normal activities of daily living (ADL). Our algorithm processes signals recorded from accelerometers built into wearable activity monitoring devices such as a Fitbit or Apple Watch that is worn on an individual's wrist. Given the potential danger of injury resulting from a fall, especially for the elderly population whom are more susceptible, an accurate fall detection algorithm could be the precursor to an autonomous emergency alert system that pages paramedics. Immediate medical intervention is critical for survival in urgent situations such as a stroke, cardiac event, or TBI; unfortunately, in many of these cases the individual may be unconscious and unable to intervene on their own behalf. With the advancement of geospatial technology, an algorithm that can distinctly detect the event of a fatal fall can automatically trigger a call for emergency medical services to the exact GPS coordinates of a mobile device or the wearable wrist device itself. We will explore the use of a combination of threshold-based and machine learning-based approaches to develop a refined fall-detection algorithm that builds upon previous research. 
%  Need to include - two to three sentences are used to state how the problem was solved. A single statement of the main result (singular) is then followed by a single statement of the main conclusion (singular).
\keywords{fall detection \and activities of daily living (ADL) \and signal processing}
\end{abstract}

% We most likely need to include a Related Works subsection to discuss previous ideas we might take when building our own algorithm.


% Note that paragraphs are created by placing a blank line before the 
% paragraph within the .tex file just as a blank line exists before the
% beginning of this comment. That blank line tells LaTeX to treat the 
% following text as a new paragraph.  No other commands are needed.


% Sections are denoted by the use of the \section{Section Name} 
% command -- where "Section Name" is the name you give to the Section.
\section{Introduction}
In 2016, approximately 30,000 adults aged 65 years and older died as the result of fatal falls in the United States, the leading cause of injury-related fatalities within this age range. The adjusted-age death rates for this senior population have steadily increased 31\% from 2007 to 2016, with an estimated 43,000 deaths due to fatal falls in 2030 if these current rates remain stable. \footnote{\url{https://www.cdc.gov/mmwr/volumes/67/wr/mm6718a1.htm?s_cid=mm6718a1_w}} These higher death rates are consistent with risk factors of advanced age as well as other associated predispositions such as: 1) reduced activity; 2) chronic conditions, including arthritis, neurological disease, and incontinence; 3) increased use of prescription medications, which act synergistically on the central nervous system; and 4) age-related changes in gait and balance.


	Globally, the World Health Organization (WHO) reports an estimated 37.3 million falls annually, the second leading cause of accidental injury-related deaths worldwide, that require medical attention and potentially hospitalization with an estimated 646,000 fatalities. \footnote{\url{https://www.who.int/news-room/fact-sheets/detail/falls}} Their findings are consistent with the fact that adults older than 65 experience the greatest incidents of fatalities from a fall, but report a wider range of risk factors that include: 1) occupations with hazardous working conditions, 2) substance abuse, 3) socioeconomic factors, 4) underlying medical conditions, 5) side effects of medication, and 6) physical depreciation. 

	% Comparison of different Fall Detection Systems: Camera Based, Environmental Sensors, Sensors worn on body

	The dangers of a fall are not life-threatening itself, but may be associated with potential complications that the elderly are especially susceptible to such as a traumatic brain injury (TBI). The critical danger of a fall is being in a prolonged supine state, in which the person remains on the ground for an extended period unable to help themselves after a fall. (Bagala et al. 2012).  This can result in severe loss of self-confidence psychologically and depending on injury, become susceptible to dehydration and internal bleeding.  The first fall detection system proposed was a personal alarm system (PAS), in which a user-activated device could be worn as a wrist band or necklace. It required the user to be conscious when a fall had occurred to press a button in order to alert emergency personnel (De Backere et al. 2015; Garripoli et al. 2015; Ozcan et al. 2017). \footnote{\url{https://link.springer.com/article/10.1007/s12652-017-0592-3}} Novel fall detection systems have since been developed to account for the possibility of unconsciousness. These systems can be categorized into camera-based systems, ambient environment sensors, and wearable sensors. This study will focus on fall detection systems using wearable sensors. We aim to develop an autonomous fall-detection algorithm that is trained on accelerometer and gyroscope data extracted from wearable activity monitors on an individual's wrist. With the detection of a fall in cases where an individual is unable to assist themselves, the wrist device would be programmed to alert paramedics and dispatch them to the exact GPS location of the device for immediate medical intervention. 
	
	Sensor placements are recommended to be placed at the head and waist position on the body for efficiency purposes when developing algorithms. While the waist intuitively makes sense as the optimal placement being the site closest to the human anatomy's center of gravity, placement at the head has shown to produce superior impact detection sensitivity. Triaxial accelerometer data from both sites produced efficient fall detection algorithms with a sensitivity around 97\% and specificity of 100\%, even with simple threshold-based algorithms. \footnote{\url{https://www-sciencedirect-com.proxy.libraries.smu.edu/science/article/pii/S096663620800026X}} Due to the limitations with comfortability and daily use of sensor placements at the head and waist, we will primarily focus on wrist sensors for its minimal invasiveness and potential for daily application. Additionally, wrist sensors expands this application beyond the targeted elderly population because wrist-worn devices can be accessorized by individuals in any age group and offer a safety buffer in the event of an accident. 
	
	% Science Direct footnote is actually a research article - using as placeholder but will change later.
	% Ethics of testing elderly patients: the experiments and activity are more taxing for them compared to relatively younger individuals. Could be a possible explanation for why datasets do not heavily test on that specific age group despite that being the targeted population for applicable use.

	We will analyze how competing algorithms perform when combining a threshold-based model with different machine learning-based models. We are specifically interested in seeing how linear discriminant analysis (LDA), support vector machines (SVM), naïve Bayesian (NB), decision trees (DT), and random forests (RF) compare on a binary-classification task that differentiates between a fall and an ADL. We will compare competing models based on the classification accuracy, specificity, sensitivity, and precision using ROC-AUC curve to visualize the results. 

	Detecting falls presents a myriad of technical, logistical, and practical challenges that we will need to address. From a modeling standpoint, an ADL-based dataset will have a very unbalanced response. This will also include a number of false flag activities, such as sitting or laying down. Additionally, falls can occur from multiple positions, such as standing, sitting, or many positions in between. This creates potential for both Type 1 and Type 2 error, and our model will need to be robust to both.
    
    ADLs encompass a wide variety of activities, categorized into six different areas: bathing, ambulatory, transferring, toileting, eating, and dressing. These types of activity will inform our majority response and will be highly dominant in the data. Right away we can tell several of these pose a risk for falls, such as ambulatory or transferring. Transferring is defined as transitory periods from one activity to another, such as getting out of bed, sitting down to eat, getting in and out of chairs or vehicles, or going to sleep at night. Our model would need to know the difference between a transferal and a fall. This could be done by testing accelerometer data against the position in space, and check if the transition was the appropriate speed. 
    
    This solution would not encompass all falls, and could ignore some important fall patterns. One common fall behavior, particularly in senior populations, is fainting during toileting. This can present an unexpected challenge - falls from a variety of different positions. If our model were only to expect falls to present a certain change in position at a certain rate, it would be unable to detect the aforementioned fall during toileting. The idea that not all falls are created equal will inform our model so that it can handle a variety of scenarios.
    
    As mentioned before, this data will have a heavily unbalanced response, so appropriate measures will need to be taken to address this, such as downsampling and stratified cross validation. One possible method of addressing many of the previous challenges as well as dealing with the unbalanced response would be training the model for anomaly detection. This would require normalizing the variance across all ADLs, then treating that variance as the positive class of a binary response. The amount of data we have to train a positive response will be more than enough for an anomaly detection architecture.
	
	% How will we define the threshold value for a fall?
		% picture a physics problem: *HP means hyper parameter*
			% an object in free fall [normalForce_up, gravitationalForce_down (HP: weight)]
			% an object at rest on the surface [dragForce_up (HP: dragCoefficient_up) , gravitationalForce_down (HP_constant: 9.8m/s^2]
	
	% Look into reinforcement learning and see if it can us solve the problem.
	
\section{Related Work} % More Background Information
\section{Materials and Methods}
\section{Results and Discussion}
\section{Conclusion}



% ---- Bibliography ----
%
% BibTeX users should specify bibliography style 'splncs04'.
% References will then be sorted according to alphabetical 
% and formatted in the correct style.
%
 \bibliographystyle{splncs04}
 \bibliography{samplebib}


% End the Document
\end{document}














